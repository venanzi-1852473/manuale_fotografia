\section{Obiettivo} \label{sec:obiettivo}

L'obiettivo, detto anche \textit{ottica} o \textit{lente}, è lo strumento usato per catturare e incanalare la
luce verso il sensore della fotocamera, sensore che acquisirà poi l'immagine.

Tanti obiettivi hanno caratteristiche molto diverse, ma una cosa che hanno tutti gli
obiettivi è la messa a fuoco (non è del tutto vero, ma per semplicità facciamo finta di sì).
Su un obiettivo c'è una ghiera che si può girare per mettere a fuoco più vicino o più
lontano, a piacimento del fotografo; ogni obiettivo ha una distanza di messa a fuoco
minima e una massima, che \textit{solitamente} coincide con l'infinito.

Le due caratteristiche principali che contraddistingono un obiettivo sono:
\begin{itemize}
    \item[-] \textbf{lunghezza focale}
    \item[-] \textbf{diaframma} 
\end{itemize}

Nel nome di un obiettivo vengono specificate la lunghezza focale e l'apertura del diaframma; un esempio di obiettivo (un classico, una volta era praticamente l'obiettivo standard) è il \lens{50}{1.8}.\newline
Del \textbf{diaframma} se ne parlerà abbondantemente più avanti, mentre della \textbf{focale} ne parliamo ora.

La \textbf{distanza focale} indica la distanza tra due particolari punti interni di un'ottica; è
espressa in millimetri, e più questo numero è piccolo (quindi più la lente è corta) e più è
grandangolare, mentre se si allunga l'obiettivo il campo visivo inizia a restringersi,
permettendo di vedere più lontano.

L'obiettivo \textbf{standard}, quasi per definizione, è il \textit{50mm}; è considerato standard perché riproduce un'immagine in modo abbastanza fedele a come la vediamo noi umani:
possiamo immaginare i nostri occhi grossomodo come due \textit{50mm}.\newline
Diminuendo la focale si hanno obiettivi \textbf{grandangolari}, che hanno un maggiore \textit{campo di visione}; solitamente si parla di grandagoli dai \textit{35mm} in giù.\newline
Aumentando la focale, si parla di \textbf{teleobiettivi}. Se poi si raggiungono focali lunghissime, dai \textit{200-300mm} in su, si parla di \textbf{supertele}.

\nb~Non ci sono regole troppo ferree su questa divisione degli obiettivi in base alla distanza focale.
Una buona regola di massima è considerare un intervallo intorno ai \textit{50mm} come standard, dai \textit{35mm} circa, a scendere, come grandangolo, e dai \textit{75$\sim$85mm}, circa, a salire come tele.

%\nb Non ci sono regole troppo ferree sulle distanze focali, nel senso che alcune fonti parlano di teleobiettivi a partire da sopra i \textit{50mm}, altre invece solo da \textit{85mm} in su, e danno un intervallo intorno ai \textit{50mm} in cui viene tutto considerato come focale standard.

Riprendiamo il nostro \lens{50}{1.8}.\newline
Il nome significa che la focale è \textit{50mm} ed è \textbf{fissa} (in inglese le lenti a focale fissa sono anche dette \textit{prime}), cioè NON si può zoommare (per gli obiettivi zoom la nomenclatura è leggermente diversa).
Mentre \textit{f/1.8} indica che l'apertura massima del \textit{diaframma} 1.8 (per più dettagli vedi \nameref{sec:diaframma}).

Un esempio di \textbf{obiettivo zoom} è il \lens{18-55}{3.5-5.6}.\newline
La focale minima dell'obiettivo è \textit{18mm} e può zoommare fino ad un massimo di \textit{55mm}.