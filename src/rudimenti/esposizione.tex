\section{Esposizione}
Esporre un'immagine significa impostare quanta luce andremo a catturare nella fotografia.

Gli elementi che contribuiscono all'esposizione di una foto sono tre:
\begin{itemize}
    \item[-] \textbf{\nameref{subsec:shuttertime}}
    \item[-] \textbf{\nameref{subsec:diaframma}}
    \item[-] \textbf{\nameref{subsec:ISO}}
\end{itemize}

\subsection{Stop}
Prima di continuare è bene capire cos'è uno \textbf{stop}, nozione molto importante.\newline
Gli stop indicano gli intervalli dei valori di eposizione; aumentare un valore di uno stop
significa raddoppiare la luce che colpisce il sensore, così come diminuire di uno stop
significa dimezzarla.

Ci sono dei valori di stop standard che tra poco vedremo. Le reflex solitamente permettono di regolarti con intervalli di $\frac{1}{2}$ stop o $\frac{1}{3}$ stop.

I valori di stop sono abbastanza intuitivi per tempo di scatto e ISO, un po' meno per il diaframma, ma non è nulla di drammatico.


\subsection{Tempo di scatto} \label{subsec:shuttertime}
Quando scattiamo una foto il sensore cattura la luce, che viene lavorata dal processore e salvata sul support di memoria (i.e. se usiamo una normale fotocamera digitale per noi è la scheda SD).\newline
Se stiamo scattando con una fotocamera analogica il processore è diverso ma analogo. Nel capitolo apposito verrà spiegato meglio come questo processo avviene su pellicola.

Il tempo di scatto viene controllato dall'\textbf{otturatore} (in inglese \textit{shutter}); questi due termini sono usati come sinonimi per indicare il tempo di scatto, è bene tenerlo a mente.

Sulle reflex moderne si possono impostare tempi di scatto, solitamente, in un range tra $\frac{1}{4000}$s e 30s.\newline
$\frac{1}{4000}$s è un tempo molto più veloce di 30s, e farà entrare quindi molta meno luce.

Usare tempi molto \textit{veloci} fa entrare poca luce e blocca la scena, ogni movimento viene congelato; al contrario con tempi lunghi entri molta luce, e ogni movimento viene catturato. Prova a scattare una foto di 1s tenendo in mano la macchinetta, sarà tutta mossa.

Per iniziare prova a usare tempi che ti permettano di esporre correttamente l'immagine senza farla venire tutta mossa; quando poi si è capito come funzionano i tempi di scatto si possono usare in modo più creativo: tempi corti per una scena che vogliamo immortalare così com'è, o tempi più lunghi per mostrare il movimento del nostro soggetto (vedi su internet \textit{car trails} e \textit{star trails}).

I principali \textbf{stop} sono:
\[ \frac{1}{4000}s, \frac{1}{2000}s, \frac{1}{1000}s, \frac{1}{500}s, \frac{1}{250}s, \frac{1}{125}s, \frac{1}{60}s, \frac{1}{30}s, \frac{1}{15}s, \frac{1}{8}s, \frac{1}{4}s, \frac{1}{2}s \]
\[ 1s, 2s, 4s, 8s, 15s, 30s \]

Sono abbastanza intuitivi: $\frac{1}{1000}$s dura il doppio del tempo rispetto a $\frac{1}{2000}$s, quindi fa entrare il doppio della luce, e quindi ha uno stop di luce in più.\newline
Il medesimo ragionamento vale per tutti gli stop; ad esempio $\frac{1}{125}$s ha 2 stop in meno di $\frac{1}{60}$s.

Le macchinette permettono anche di andare oltre i 30s, usando la \textbf{Posa Bulb}, o \textit{posa B}.
Con la posa B possiamo decidere noi quando interrompere lo scatto, quindi possiamo decidere una durata completamente arbitraria sotto o sopra i 30s.

\nb Sulle fotocamere digitali foto a \textbf{lunga esposizione} consumano molta batteria; inoltre perdono qualità a causa del \textit{rumore} (per ora è bene tenere a mente giusto la durata della batteria)


\subsection{Diaframma} \label{subsec:diaframma}
Ogni obiettivo ha una ghiera "circolare" che si può restringere e allargare, entro certi
limiti fisici, per far entrare più o meno luce.

Il valore è indicato da un numero preceduto da $f/$; gli stop del diaframma sono detti, in inglese, \textit{f stops}.

I principali \textit{stop} sono:
\[ 1.0 1.4 2.0 2.8 4.0 5.6 8.0 11 16 22 35 45 64 \]

\nb La scala può continuare anche oltre $f/64$, ma le fotocamere che si usano al giorno d'oggi usano obiettivi che solitamente non vanno oltre $f/22$.

Funzionano al contrario, qui la controintuitività: più è basso il valore, più è aperto il
diaframma e più luce entra.\newline
Aprire a $f/2.8$ fa entrare il doppio della luce rispetto a $f/4$.

Il diaframma funzione come la pupilla dell'occhio: quando c'è tanta luce si restringe per farne entare di meno.

Il diaframma in realtà influenza anche altri due parametri importantissimi per una foto:
nitidezza e profondità di campo, ma questa cosa verrà spiegata più avanti.

\fbox{Il file originale ha un'appendice che qui voglio spostare in Rudimenti non, devo ricordarmi di farlo}


\subsection{ISO}  \label{subsec:ISO}

