\section{Sugli stop} \label{sec:sugli_stop}

Vista la teoria è ora di fare un po' di pratica con la macchinetta. Per capire veramente bene come usare gli stop di luce è importante capire il seguente concetto: uno stop di luce è lo stesso indipendentemente da dove lo prendiamo, se dal tempo di scatto, dal diaframma o dagli ISO. Quindi se togliamo uno stop di luce da una parte e lo recuperiamo da un'altra l'immagine sarà, dal punto di vista dell'esposizione, uguale.

\esperimento prendiamo dei valori con cui scattare una foto, ad esempio \fotoargs{$\frac{1}{250}$}{4}{100}; scattiamo ora una foto con i seguenti valori: \fotoargs{$\frac{1}{1000}$}{2.8}{200}.\newline
Le due foto hanno la stessa esposizione: nella seconda foto abbiamo tolto due stop di luce dal tempo di scatto, poi ne abbiamo recuperato uno con il diaframma e uno con gli ISO.

Prove come questa possono essere fatte con qualunque valore, si può sperimentare con diversi valori per capire bene gli stop e prenderci familiarità.