\section{Conclusione capitolo} \label{sec:conclusione_rudimenti}
È importante studiare e capire per bene questo capitolo, sono pochi concetti che si imparano in poco tempo, per padroneggiarli ci vuole un po' di tempo, ma capirne la logica dietro è semplice e non ci vuole molto tempo.

Alla fine di questa introduzione, una nota per rimarcare il concetto: capire gli elementi che contribuiscono all'esposizione di una foto (aiutandosi con l'esposimetro) è importantissimo, sia per capire cosa si sta facendo, sia per poter poi prendere decisioni più creative.
È bene ribadirlo, l'esposimetro non è un oracolo sceso in terra, non serve premurarsi che la tacchetta sia al centro, e che quindi secondo l'esposimetro l'immagine sia esposta correttamente;
l'esposimetro può non essere sempre corretto al 100\% e può non rispecchiare quella che è la nostra idea, l'immagine finale che vogliamo avere. Può capitare che in una scena vogliamo sottoesporre un po' l'immagine per catturare tutti i dettagli nelle nuvole, ma l'esposimetro cerca di farci esporre l'immagine con 1-2 stop in più per bilanciarla, in questo caso possiamo usare l'esposimetro come punto di riferimento e poi sottoesporre un po' la foto per ottenere il risultato che abbiamo in mente.