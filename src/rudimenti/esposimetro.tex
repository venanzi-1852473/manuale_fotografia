\section{Esposimetro} \label{sec:esposimetro}
Ora che abbiamo visto quali sono i valori da impostare, e in che modo incidono sulla foto, rimane un grande problema: come impostarli?\newline
Con un po' di pratica si riesce a capire quali valori ci danno una foto esposta correttamente, ma certamente non dobbiamo stare a scattare diverse foto fino a che non troviamo una combinazione adatta; arriva in nostro soccorso l'\textbf{esposimetro}.

L'esposimetro, come suggerisce il nome, ci dice come esporre l'immagine. Per noi è una tacchetta che ci dice se l'immagine è esposta correttamente, se è \textbf{sovraesposta} (i.e. troppa luce) o se è \textbf{sottoesposta} (i.e. poca luce).\newline
Gli esposimetri sulle fotocamere moderne ci dicono se l'immagine è esposta correttamente con un margine di $\pm$3 stop, ovvero se è sovraesposta fino a 3 stop di luce o sottoesposta fino a 3 stop. Fotocamere più vecchie hanno una tolleranza di 2 stop.

Inoltre l'esposimetro ci dà questa informazione con una precisione di $\frac{1}{3}$ stop. Cosa significa questo? Quando settiamo la fotocamera non dobbiamo saltare tra uno stop e l'altro, sempre raddoppiando o dimezzando la luce, possiamo anche settare valori intermedi; ad esempio tra $\frac{1}{200}s$ e $\frac{1}{100}s$ la fotocamera ci può far impostare anche $\frac{1}{160}s$ e $\frac{1}{125}s$, che sono valori intermedi tra i due stop di luce.
L'esposimetro ci aiuta informandoci anche di queste piccole variazioni tra due stop di luce.

È importante però capire che l'esposimetro non è infallibile, per diversi motivi. Primo fra tutti non è detto che vogliamo scattare un'immagine sempre esposta in modo corretto, neutro; inoltre ci sono diversi metodi con cui l'esposimetro valuta la scena, e il metodo impostato potrebbe non essere quello che a noi serve (i metodi di misurazione verranno spiegati meglio in \fbox{aggiungere ref}).