\chapter{Fotografia analogica} \label{ch:analogica}

La prima fotografia della storia risale al 1826, scattata da Nicèphore Niepce.
A seguire negli anni ci sono state numerose scoperte e tanti sviluppi, che hanno portato la fotografia ad essere sempre più fruibile.

I primi tentativi di costruire una fotocamera con un sensore digitale avvengono nella seconda metà del XX secolo, e hanno in seguito portato ad una diffusione della fotografia digitale, su larga scala, all'incirca con il coincidere del III millennio.

Le fotografia analogica non solo è quindi stata la scelta primaria di molti fotografi fino a poco più di 20 anni fa, ma per ben oltre un secolo è stata l'unica scelta possibile.

Cambia tanto dalla fotografia digitale? Dipende da quale punto di vista, certamente il modo di archiviare le foto, così come la post produzione, sono totalmente cambiati, ma la fisica non varia, tante delle cose che in questo manuale sono state delle riguardo le fotocamere digitali valgono anche per le fotocamere analogiche, spesso con la sola differenza che la luce non viene catturata da un sensore ma impressa su una pellicola.

\pagebreak