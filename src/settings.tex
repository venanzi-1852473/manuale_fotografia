\graphicspath{ {../images/} }

\hypersetup{
    colorlinks,
    citecolor=black,
    filecolor=black,
    linkcolor=blue,
    urlcolor=black
}

%per rimuovere l'header dai capitoli
\titleformat
    {\chapter}[display]
    {\normalfont\bfseries}
    {}
    {0pt}
    {\LARGE}

%\titleformat
%    {\section}[display]
%    {\normalfont\bfseries}
%    {}
%    {0pt}
%    {\Large}

\newenvironment{Figure}
  {\par\medskip\noindent\minipage{\linewidth}}
  {\endminipage\par\medskip}


%rende più alte le righe delle tabelle (serve per far entrare bene le frazioni)
\renewcommand{\arraystretch}{2}

%titolo della table of contents
\renewcommand{\contentsname}{Indice}

\pagestyle{plain}
\fancyhf{}

\rfoot{pag. \thepage}

\addbibresource{bibliography.bib}

%copiato da https://tex.stackexchange.com/questions/77809/how-do-i-remove-white-space-before-or-after-the-figures-and-tables
\setcounter{topnumber}{2}
\setcounter{bottomnumber}{2}
\setcounter{totalnumber}{4}
\renewcommand{\topfraction}{0.85}
\renewcommand{\bottomfraction}{0.85}
\renewcommand{\textfraction}{0.15}
\renewcommand{\floatpagefraction}{0.8}
\renewcommand{\textfraction}{0.1}
\setlength{\floatsep}{5pt plus 2pt minus 2pt}
\setlength{\textfloatsep}{5pt plus 2pt minus 2pt}
\setlength{\intextsep}{5pt plus 2pt minus 2pt}

%cambia il caption delle immagini con "fig.", invece del "Figure" di default
\renewcommand{\figurename}{Figura}
\renewcommand{\tablename}{Tabella}

