In queste pagine si cercherà di dare delle nozioni di base che siano sufficienti per iniziare a capire come scattare con la fotocamera in manuale.
Dopo aver visto le basi di una \textit{corretta} esposizione, verranno date altre nozioni che servono per capire meglio cosa si ha in mano quando si scatta, oltre ad avere altri strumenti per poter prendere decisioni più creative e meno statiche.

Il manuale è diviso in tre capitoli:
\begin{itemize}
    \item[-] \textbf{\nameref{ch:rudimenti}}: contiene le informazioni di base per capire come esporre correttamente una foto, più poche altre nozioni molto importanti
    \item[-] \textbf{\nameref{ch:rudimenti_non}}: contiene informazioni non necessariamente più complicate di quelle del capitolo precedente, sono informazioni che all'inizio possono essere glissate. Questo capitolo spiega più nel dettaglio gli strumenti che usiamo, in modo tale da avere più consapevolezza di cosa usiamo quando scattiamo per poi avere più possibilità nel momento in cui andiamo ad azionare l'otturatore
    \item[-] \textbf{\nameref{ch:analogica}}: un accenno alla fotografia analogica, per chi vuole tentare o per chi vuole capire cosa c'era prima del sensore digitale
\end{itemize}

Questo manuale vuole essere solo un primo passo nella fotografia, con lo scopo di insegnare a scattare le prime foto e di dare una panoramica generale sulla fotografia.
Tanti aspetti non sono coperti (come ad esempio gli accessori, i flash, i filtri, ed altri concetti che un fotografo dovrebbe quantomeno conoscere), mentre tutti gli altri non vengono approfonditi nel migliore dei modi.

Una volta letto questo manuale si dovrebbero, sperabilmente, avere le conoscenze e le competenze per iniziare a muoversi da soli nel mondo della fotografia.
Per voler conoscere e/o approfondire altri argomenti consiglio di comprare un manuale in libreria, ne esistono tanti e per tutte le esigenze; sono scritti da fotografi con anni di esperienza e hanno dietro una redazione di persone che si occupano, per lavoro, di far sì che i libri vengano scritti bene, in tutti gli aspetti.

Gran parte delle informazioni presenti in questo manuale vengono dai due manuali sui quali ho studiato le basi della fotografia:
\begin{itemize}
    \item[-] \cite{gatcum2017manuale} per quanto riguarda la fotografia più in generale
    \item[-] \cite{marquardt2020fotografia} per quanto riguarda la fotografia analogica
\end{itemize}

Se qualche informazione viene da altre fonti verrà adeguatamente segnato.

\begin{center}
    \fbox{
        \parbox{\linewidth}{
            Nota: il manuale è per lo più completo, ma potrebbero mancare alcune citazioni, così come alcuni accorgimenti per rendere il manuale più fruibile (come ad esempio la presenza di immagini o la formattazione delle immagini stesse)
        }
    }
\end{center}