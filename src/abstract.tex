In queste pagine si cercherà di dare delle nozioni di base che siano sufficienti per iniziare a capire come scattare con la fotocamera in manuale.
Dopo aver visto le basi di una \textit{corretta} esposizione, verranno date altre nozioni che servono per capire meglio cosa si ha in mano quando si scatta, oltre ad avere altri strumenti per poter prendere decisioni più creative e meno statiche.

Il manuale è diviso in tre capitoli:
\begin{itemize}
    \item[-] \textbf{\nameref{ch:rudimenti}}: contiene le informazioni di base per capire come esporre correttamente una foto, più poche altre nozioni molto importanti
    \item[-] \textbf{\nameref{ch:rudimenti_non}}: contiene informazioni non necessariamente più complicate di quelle del capitolo precedente, sono informazioni che all'inizio possono essere glissate. Questo capitolo spiega più nel dettaglio gli strumenti che usiamo, in modo tale da avere più consapevolezza di cosa usiamo quando scattiamo per poi avere più possibilità nel momento in cui andiamo ad azionare l'otturatore
    \item[-] \textbf{\nameref{ch:analogica}}: un accenno alla fotografia analogica, per chi vuole tentare o per chi vuole capire cosa c'era prima del sensore digitale
\end{itemize}