\section{Nomenclatura} \label{sec:nomenclatura}

In questa sezione vediamo alcuni termini che ritornano spesso, per capire meglio di cosa stiamo parlando. Su ognuno dei seguenti termini si potrebbe scrivere un intero capitolo dedicato, qui vediamo un accenno.

\subsection{Megapixel} \label{subsec:megapixel}
Indica il numero di pixel in una foto. Il prefisso \textit{mega} significa $10^6$, quindi 1 megapixel è un milione di pixel ($10^6 = 1\,000\,000$).
Molte fotocamere digitali al giorno d'oggi hanno sensori da 24 megapixel, ovvero scattano foto con $24 \cdot 10^6 = 24\,000\,000$ pixel. 24 megapixel è solo uno dei tanti formati, esistono sensori più
piccoli in termini di pixel e sensori molto più carichi, con addirittura sensori sui 100 megapixel.

Il termine megapixel è usato come sinonimo di \textit{risoluzione}, ma quest'ultimo è usato anche per indicare i \textit{DPI - Dots Per Inch} o i \textit{PPI - Pixels Per Inch} di una foto, è importante non confonderli.


%\subsection{Angolo di campo}  \label{subsec:angolocampo} 
\subsection[Angolo di campo]{Angolo di campo \footnote{Nota dell'autore: in~\cite{gatcum2017manuale} viene chiamato Angolo visivo, nome che però non ho trovato da nessun'altra parte; probabilmente si tratta solo di una traduzione in italiano poco felice. In ogni caso si tratterebbe di un altro nome per indicare lo stesso concetto.}} \label{subsec:angolocampo}

È un concetto collegato alla focale di un obiettivo e alla grandezza del sensore (vedi \nameref{sec:obiettivi_approfondimento} e \nameref{sec:sensore}).

Indica la porzione di scena che può essere presa dall'obiettivo, ed è espresso in \textit{gradi}. Immaginiamo di avere un angolo posto davanti la lente, con la punta dell'angolo che tocca la l'obiettivo,
l'angolo di campo sono i gradi di questo angolo.

Se un sensore full frame (vedi \nameref{subsec:sensorifullaps}) un 50mm ha un angolo di 46.8º. Più un obiettivo è grandangolare maggiore è la scena che può riprendere, e quindi più largo sarà questo angolo, il contrario accade se l'obiettivo diventa più lungo (i.e. aumentando la focale). Quindi tornando al sensore full frame, un 28mm ha un angolo di campo di 75.4º, mentre un 135mm 18.2º.

I valori di cui sopra sono solo esempi, non serve neanche impararli, sono solo esempi fatti per far capire il concetto.


\subsection{Boke} \label{subsec:boke}
Anche scritto \textit{bokeh}, è una parola giapponese e significa \textit{sfocatura}.\newline
Si usa per indicare lo sfondo sfocato, spesso presente nei ritratti e in particolare nelle foto macro.


\subsection{Tonalità} \label{subsec:tonalita}
Tonalità non è altro che sinonimo di \textit{colore}.\newline
In fotografia si parla di:
\begin{itemize}
    \item[-] \textbf{Toni alti}: bianco e tutti i colori così luminosi da avvicinarsi al bianco
    \item[-] \textbf{Toni bassi}: nero e tutti i colori così scuri da avvicinarsi al nero
    \item[-] \textbf{Mezzi toni}: tutto il resto che c'è tra toni alti e bassi   
\end{itemize}


\subsection{Vignettatura} \label{subsec:vignettatura}
Un'immagine presenta vignettatura quando gli angoli sono più scuri rispetto al resto dell'immagine.
La vignettatura può essere aggiunta in un secondo momento, è una scelta artistica, mentre in molti casi può essere un problema della lente: su alcune lenti è più evidente di altre, ma in generale tutte le lenti
tendono ad avere un po' di vignettatura con il diaframma completamente aperto, mano mano che si chiude il diaframma la vignettatura sparisce.


\subsection{Aberrazione cromatica} \label{subsec:abberrazione}
È un difetto delle lenti che compare nelle foto. L'aberrazione cromatica avviene quando la lente non riesce a mettere a fuoco tutti i colori nello stesso punto, l risultato sono \textit{sbavature} colorate.

L'aberrazione cromatica avviene spesso nei punti ad elevato contrasto; i.e. sul confine tra una zona molto scura e una molto chiara.
Può manifestarsi molto anche gli angoli delle foto, dove le immagini spesso diventano più morbide.

Per evitare, o rimuovere, l'aberrazione cromatica si può, oltre che fare attenzione alla propria inquadratura (ad esempio evitando di mettere, se possibile, il nostro soggetto direttamente davanti una forte fonte di luce),
chiudere il diaframma oppure rimuoverla il digitale in un secondo momento.

L'aberrazione cromatica compare spesso di colore viola o verde.


\subsection{Distorsione ottica} \label{subsec:distorsioneottica}
Gli obiettivi non mostrano le immagini così come appaiono nella realtà, bensì vengono sempre un po' distorte.

Obiettivi grandangolari tengono a causare una distorsione a \textbf{barile} \label{def:distorsionebarile}, dove l'immagine sembra più bombata.
I teleobiettivi invece fanno il contrario, causano una distorsione a \textbf{cuscino}\label{def:distorsionecuscino}, l'immagine viene risucchiata verso il centro, come se ci fosse un peso che attira tutto a sé.