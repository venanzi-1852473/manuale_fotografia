\section{Punti di messa a fuoco} \label{sec:puntimessafuoco}
I punti di messa a fuoco sono uno strumento di grande aiuto; permettono di fare foto con molta più facilità e con risultati più consistenti.

Quando guardiamo dentro il mirino di una fotocamera notiamo alcuni puntini neri sparsi nel mirino, sono i \textit{punti di messa a fuoco}.
Il numero di punti dipende dalla fotocamera. Le prime fotocamere a montarli avevano un singolo punto di messa a fuoco al centro del mirino; addirittura alcune fotocamere analogiche, come la Canon 1000F, hanno un punto di messa a fuoco. Oggi ci sono modelli con una manciata di punti, fotocamere di fascia medio-bassa ne hanno 9 o poco più, mentre fotocamere più professionali ne possono avere decine, se non centinaia, o nelle mirrorless addirittura migliaia.

A cosa servono? Come suggerisce il nome servono per mettere a fuoco.
Quando usiamo la modalità autofocus (i.e. messa a fuoco automatica) la fotocamera cerca di mettere a fuoco sui punti di messa a fuoco.
Quando la fotocamera ritiene di aver messo a fuoco i punti da neri di illuminano di rosso.

Quali punti di messa a fuoco usa la fotocamera?\newline
Si possono impostare, e le fotocamere supportano almeno queste quattro modalità:
\begin{itemize}
    \item[-] \nameref{subsec:multiarea} o \textbf{Valutativa}
    \item[-] \nameref{subsec:parziale}
    \item[-] \nameref{subsec:ponderata} o \textbf{Pesata centrale}
    \item[-] \nameref{subsec:spot}
\end{itemize}

Fotocamera con molti punti di messa a fuoco permettono misurazioni più flessibili, ma nella maggioranza dei casi le modalità a seguire sono più che sufficienti.
Ricordiamo inoltre che, sebbene l'autofocus sia un grande aiuto, specialmente quando vogliamo fotografare scene molto veloci e dinamiche, non è affatto necessario per scattare, esiste sempre la messa a fuoco manuale, che in certi casi è ancora d'obbligo.

\subsection{Multi area} \label{subsec:multiarea}
È l'impostazione di base. La fotocamera usa tutti i punti di messa a fuoco per valutare quali usare e quindi dove mettere a fuoco.

È da usare con attenzione, specialmente in scene con grandi differenze di tonalità (i.e. zone molto chiare e zone molto scure), la fotocamera può fare fatica a mettere a fuoco la zona di nostro interesse.


\subsection{Parziale} \label{subsec:parziale}
Viene valutata solo uan piccola zona al centro dell'inquadratura. Funziona molto bene se il soggetto è al centro della nostra foto.

\subsection{Ponderata centrale} \label{subsec:ponderata}
Funziona come la \nameref{subsec:multiarea}, ma viene data più importanza al centro della foto.

\subsection{Spot} \label{subsec:spot}
Permette di selezionare manualmente i singoli punti di messa a fuoco; così possiamo puntare un'area molto limitata dell'intera inquadratura, all'incirca 1-5\%.

È il metodo che sicuramente richiede più attenzione da parte della persona, ma al contempo è quello più affidabile, che ci lascia scegliere esattamente quale punto vogliamo mettere a fuoco.


\subsection{Esposizione} \label{subsec:messafuocoesposizione}
I punti che vengono usati per mettere a fuoco, sono usati anche dall'\nameref{sec:esposimetro} per valutare l'esposizione della scena.

In questo senso, come già accennato in precedenza, l'esposimetro non è sempre affidabile. Se ad esempio usiamo una valutazione Multi area, e ci sono grandi aree illuminate e grandi aree molto scure nella foto, l'esposimetro potrebbe fare fatica a capire su quale parte vogliamo concentrarci e potrebbe fare una lettura non adeguata della scena.

Questi problemi possono essere sia mitigati con l'esperienza, sia usando metodi di valutazione più mirati, come ad esempio lo \nameref{subsec:spot}, che limita molto l'area da analizzare.


\subsection{Messa a fuoco manuale} \label{subsec:puntifuocomanuale}
Le fotocamere permettono di usare i punti di messa a fuoco anche con la messa a fuoco manuale, dovremo essere noi a mettere a fuoco, ma l'esposimetro valuta la scena usando i punti selezionati e i punti si illuminano comunque di rosso quando li abbiamo messi a fuoco.